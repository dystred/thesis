This thesis studies practical applications of the Greeks-free hedging methods known as deep hedging, and in particular the performance of said hedging method in real-world and controlled trading environments with model uncertainty. As in \cite{buehler2019deep}, I find the method to work well in an empirical application to vanilla SPX options. However, the analysis provides several additional perspectives. In particular, I find that deep hedging does not \emph{unconditionally} outperform classic Greeks-based approaches when hedging vanilla call options in a real trading environment. This stands in contrast to what one would observe in essentially any fixed incomplete model setting 
%and contrary to the impression one might get based on the aforementioned article.  
%Including a richer set of classic Greeks-based hedging strategies based on the Black \& Scholes and the Heston models that leverage implied volatility information from the options market. This empirical finding is supported by several controlled simulated experiments in which I introduce model uncertainty, first in a constant volatility model and subsequently in the stochastic volatility environment of the Heston model. 

First, I show that deep hedging is significantly more vulnerable to an unexpectedly high volatility level in the constant volatility model. Second, I exhibit in a controlled Heston stochastic volatility environment the consequences of deep hedging under a variance process with parameters different from the ones governing the physical market dynamics. I find that if one goes ahead and trains the deep hedging strategy to minimize the 75\% conditional value at risk, getting the level of the variance wrong has substantially smaller consequences than getting its volatility wrong if the trader's preferences are well described by the training objective. An important implication of this finding is that one does not significantly harm the efficiency of deep hedging by ignoring the fact that a negative variance risk premium typically exists. In other words, calibrating the Heston model to options data and subsequently training the deep hedging model on market scenarios generated under the inferred equivalent martingale (pricing) measure is no disaster when the trader's preferences are well described by the training objective.%. This conclusion is, however, reversed if the trader cares more about extreme tail risk than indicated by the objective under which she trained the deep hedging model.

In the empirical application of deep hedging to plain vanilla SPX call options, I find, in line with the controlled experiments, that the performance of deep hedging suffers when volatility suddenly spikes during the extreme market scenario of the Covid crisis. During this time, the neural network policy sometimes diverges due to not having seen such extreme figures during the model training phase. However, any serious practical implementation of the model would be more robust than the one considered in the analysis of this thesis. In any case, this finding at least highlights the demand for implementing, along with the deep hedging model, some type of sanity filter or to robustify the model by oversampling extreme scenarios during model training. 

I find some evidence that using a Realized GARCH model for market simulations as opposed to the GJR-GARCH model performs better during sudden volatility spikes, attributable to its steeper news impact curve. Additionally, the empirical analysis shows that hedging with a model trained on simulated Heston stochastic volatility paths inferred from the pricing measure is a viable alternative to training the model under the historical measure. This is in line with the findings from the controlled market environment. This is an important finding for practitioners, among which the Heston model is popular. Furthermore, it extends on the existing literature in \cite{giurca2021delta} and \cite{mikkila2021empirical} who also test empirically the performance of deep hedging when the model is trained under a Heston model calibrated to the options market.
%useful for decreasing tail risk compared to both conventional Greeks approaches and deep hedging models trained under the physical dynamics. This is an important finding for practitioners, among which the Heston model is popular. It also extends on the existing literature in \cite{giurca2021delta} and \cite{mikkila2021empirical} who also test empirically the performance of deep hedging when the model is trained under a Heston calibrated to the options market.

In the empirical test, I find, somewhat contrary to what one might expect, that the gains from switching from a Greeks-based hedging strategy to deep hedging are not always increasing in the degree of market incompleteness, as measured by the extent of time discretization between hedge rebalancing. 

Finally, it is important to keep in mind that the analysis presented here has disregarded all types of trading costs and frictions, except the discretization of time - the reason being that it requires various ad-hoc adjustments to standard Greeks hedging methods to hedge efficiently under these conditions. On the other hand, accommodating such real-life frictions in the deep hedging strategy requires only a trivial adjustment of the profit-and-loss calculation during model training. Combined with the results presented throughout the thesis, this suggests that the efforts on practical implementations of the modern deep hedging strategy in larger financial institutions is going to continue.

While this thesis has provided some insights for the practical relevance of deep hedging for vanilla options hedging, there is still substantial work to be done. In particular, from the practical perspective also taken in this thesis, it might be interesting to explore other ways to incorporate high-frequency based trading information in the form of trading volumes or other realized measures. Deep hedging theoretically allows the trader to exploit a large information set when deciding on the strategy, but a question still to be answered is how one should incorporate information such as news. In practice, one must formulate a dynamic stochastic model for such alternative variables, at least with the approach taken in this thesis. An alternative to formulating a stochastic model for the market is to simply train the deep hedging model on historical data, thus making the strategy truly data-driven. This is the approach in \cite{mikkila2021empirical}. In such a setting, incorporating alternative information appears much more straightforward, thus really exploiting the flexibility of deep hedging. Research along those lines would be highly interesting and relevant in practice.


%A substantial advantage with deep hedging over classic methods is its immense flexibility. The model simply learns to hedge well on the example market scenarios that has been presented during model training. An essential determinant of the real-life performance of deep hedging is thus the choice of \emph{market generator}, which should be chosen to model the stochastic dynamics of relevant market variables as well as possible. In the baseline deep hedging model implemented in the empirical part of the thesis, market scenarios are generated with the GJR-GARCH model of \cite{glosten1993relation}. I try to exploit the flexibility and test whether the empirical deep hedging performance can be improved by leveraging more accurate volatility information based on either high-frequent observations on intra-daily prices from the statistical measure or by using the volatility information from the options pricing measure used by the market. To this end, I first estimate the Realized GARCH model of \cite{hansen2006realized} and use that model for scenario generation. Second, I analyze how deep hedging performance changes when markets are instead simulated using the Heston stochastic volatility model, daily recalibrated to fit the options smile, i.e. under the \emph{pricing} measure. I find that including high-frequency-based volatility information does not increase hedging performance. On the other hand, I find that the deep hedging model trained in the stochastic volatility environment reduces the risk of large losses, however, at the cost of a greater average loss, compared to the equivalent model trained under the statistical measure. Interestingly, the hedging performance of said deep hedging model is quite similar to the locally risk-minimizing Heston delta-vega hedge in from \cite{poulsen2009risk}.

%Third, I exhibit in a controlled Heston stochastic volatility environment the consequences of deep hedging under a CIR variance process with parameters different from the ones governing t  he physical market dynamics. I find that if one goes ahead and trains the deep hedging strategy to minimize the 75\% conditional value at risk, getting the \emph{level} of the variance wrong has substantially smaller consequences than getting its \emph{volatility} wrong if the trader's preferences are in fact well described by the training objective. An important implication of this finding is that one does not harm significantly the efficiency of deep hedging by ignore the fact that a negative variance risk premium typically exists. In other words, calibrating the Heston model to options data and subsequently training the deep hedging model on market scenarios generated under the inferred equivalent martingale (pricing) measure is not a \emph{disaster}. This conclusion is, however, reversed if the trader cares more about extreme tail risk than indicated by the objective under which she trained the deep hedging model.