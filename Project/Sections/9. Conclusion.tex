%%%% CONCLUSION

Setting out to investigate the relation between the implied volatility spread and stock returns, I expected to find the signal to be priced in the cross section, and the portfolios to provide a clear pattern in the returns. I expected the returns of the tail portfolios to be more extreme and the returns of the middle portfolios to be fairly similar. 

Stock returns and option prices are available at a daily granularity through CRSP and OptionMetrics. The sample spans from 1996 to 2021, and consists of all stocks registered at American exchanges which have an option written upon them and traded within the last 7 days. Combining this data into portfolios allows me to remove idiosyncratic noise from the weekly returns. Choices regarding the formation of the portfolios are all highlighted, and alternative results are shown in the appendix to show the robustness of the analysis. The implied volatility spread is mutated into two distinct signals. First is the value of the implied volatility spread of the latest trading day before portfolio formation, and the second is the change within the last 7 days in the implied volatility spread. The first signal estimated as a proxy for the beliefs of the market towards the future distribution of the stock, and the second signal a proxy for new information affecting a change in the expectations to the future stock price.

The cumulative returns of the portfolios are clearly positively correlated with the value of the implied volatility spread. In the unconditional analysis, I find the portfolios' returns to show a clear monotonic relationship, and the long-short factors not to be fully spanned by other external factors. This indicating that informed investors participate in the option markets before moving over to the stock market, resulting in private information affecting the prices in the option markets first.

Alternating the choices made for the portfolio formation, I find that across the amount of portfolios, the amount of sortings, and the choice of signal, the monotonic relationship persists. The cumulative returns of the portfolios increases in the value of the implied volatility spread. Intuitively, this means that a high call price indicates higher probability of the stock increasing in value. This makes intuitively sense, as a high implied volatility spread means a higher priced call, insinuating that the investors believe the stock to increase in value.

In the conditional analysis, the factor competition shows that the long-short net-zero investment factors in the portfolios are not redundant when comparing with Fama \& French's five factor model, and when correcting for a subjectively chosen list of factors from Open Asset Pricing. This could be due to the factor proxying investors' private information or it could be due to shorting limitations in the stock market, resulting in investors taking positions in the option markets to get their desired exposure.
%stimating the price of risk through both Fama-Macbeth regressions and the three-step procedure, the analysis provides no apparent results, even when accounting for time-varying risk premia and other external factors. 
I am also using both Fama-macbeth regressions and the three-step procedure to evaluate the price of risk. The Fama-Macbeth regressions provide inconclusional results, using either time invariant or time varying exposure, and accounting for just the factor or including all five factors from \cite{fama2015five}. The three-step procedure did provide a small risk premia of the factor, but the explanatory power of the model was very low, so the evaluation of the factor strength was not relevant. 
A pooled cross sectional regression provides only the expected relation between the portfolios, and intuitive coefficients for the included factors. 
%The last part of the conditional analysis was a pooled cross sectional analysis, with dummies for the portfolio indexes and a varying amount of factors included. The results gave a clear distinction between the average results of the different portfolios, and the coefficients of the external factors included were intuitive. 

The economic analysis of the portfolios are based on the optimal portfolio allocation of a rational investor, without using the utility framework. When forming the efficient frontier and the optimal allocation given allocation constraints, the portfolios formed on the implied volatility spread were excluded in favor of including the size and value portfolios of Fama \& French. Only in the case of imposed allocation constraints were the portfolios allocated a positive weight. Thus in a mean-variance framework, the portfolios seem to be deprecated in favor of other external portfolios.

Using the portfolio optimisation framework, we are assuming the primary risk factor of concern to the investor to be the variance. The implied volatility spread does, however, catch the loss aversion of the investors, as puts are often overpriced. It would make sense to adapt the portfolio formation to take this factor into account, and include thoughts from prospect theory to evaluate the optimal portfolio.
% It would make sense to take this loss aversion into account when forming the optimal portfolio in the mean-variance framework. 

The approach deployed here is subject to intense data snooping, as there is no formal and correct way prespecified. Therefore, I have strived to illuminate all choices made, and provide sufficient alternative results throughout the analysis and in the appendix.

I believe that further research into the maturing of the option markets would yield more nuanced results, compared to the results supplied in this project spanning the entire sample period. Furthermore, it could provide relevant insights, if the signal were adapted to incorporate only the negative spreads, so an analysis was conducted upon the signal of $\left(IV_{call}-IV_{put}\right)^{+}$, to proxy for the risk associated with a put being overpriced. This could also be proxied by including the PIN factor, which measures the ratio of buyer-initiated calls and puts, to identify the informed investors and their trading.

In addition to the risk premia estimated through Fama-Macbeth regressions and the three-step procedure, a machine learning appproach for estimating the SDF on the basis of this new factor, would be relevant. This would allow for a mapping between the $\mathbb{P}$ and $\mathbb{Q}$ measure as discussed in the option pricing section.

All in all, I find that the rational investor would not invest in the portfolios, if it were among the ones considered for the efficient frontier, but it does provide a small weekly premium according to the three-step procedure. In general, the results are stable across specific choices made in the signal and portfolio formation, and it seems to be providing some variation not explained by other external factors, which leads me to assume that I have found a somewhat relevant factor of the cross section of stock returns.
