%%%% SETTING THE STAGE

The setting of this study is elaborated in this section. I will start out by explaining the implied volatility spread, how it is derived and why it is a relevant signal. Then I will go on to discuss the ideas underlying the factor models and how they came to be through the arbitrage pricing theory. Next the rational investor's preference will be discussed to support the analysis of optimal portfolio formation, and subsequently a few concepts from prospect theory will be outlined to give some nuance to discussion. Lastly I will include a small note on the statistical properties of the data, in regards to problems arising from heteroscedasticity, autocorrelation and stationarity.

\subsection{Option Pricing}

The interest of this entire project is centered around the implied volatility spread of american options on stocks across the American stock market within the last 28 years. 

The optionmarket is inherently forward looking, as options are contracts for future differences between the strike and stock price. In general, the price of a call option with time to maturity, $T$, and strike, $K$, is defined as follows:
\begin{equation}
	C_{t}(K,T)=\mathbb{E}_{t}^{\mathbb{Q}}\left[\left(S_{T}-K\right)^{+}\right]
\end{equation}
where the right hand side of the equation just shows the expectation under the $\mathbb{Q}$-measure\footnote{The $\mathbb{Q}$ denotes risk-neutral pricing and will be elaborated upon further in the following subsections.} to the positive difference between the stock price, $S_{T}$, at maturity and the strike while assuming the risk free rate, $r$, to be zero (and constant). As an option is a contract for difference, where the buyer chooses if they want to exercise the contract for the difference at prespecified moments before maturity, they provide an alternative to taking an outright position in the underlying for an investor with different views (or information) than the market. In addition to exposing the investor towards this outright stock price, the non-linearity of the option pay-off also allows the investor to take exposure in the volatility of the stock. 

The optionmarkets are characterised by trading a lot of different european options upon indexes and the like, while the options on individual stocks are american. These two kinds of options share many similarities, and I will therefore start by outlining the theory for pricing an european option and then dive into the differences and how this affects the prices of american options. The focus within these classes of options will be both puts and calls. I will refrain from including any other kinds of options, as their pricing and payoff structure is different, and therefore might affect the results in an unintended way. 

Given the historic context of the American stock market increasing in value over the last decades, it seems natural that an investor will expect the future value of a stock to be higher than the current value. Therefore, one would conclude that calls should be priced higher than puts, if the strike is equal to the current traded price. However, some investors use the puts to lock in their profits, as they secure a future salesprice at the strike level of the put. And according to prospect theory\footnote{See subsection \ref{subsection:prospect}.}, this loss aversion is rational and very present in all financial markets.

With a put and a call with the same strike, maturity and underlying, it is natural to conclude that the prices of these two products should be related. This is defined as the put-call parity, and it holds only for european options:
\begin{equation}
	C_{t}\left(K,T\right)+Ke^{-r\left(T-t\right)} = P_{t}\left(K,T\right)+S_{t}
\end{equation}
where the price of a call in addition with the present value of the strike (discounted at the risk-free rate) is equal to the sum of the put price and the current stock price. This relation holds as a position in a call option and lending at the risk free rate with a payoff of $K$ at time $T$, against a position in one put option and the share, results in the exact same payoff at time $T$. This should be priced the same according to the law of one price. This also means that one can replicate the payoff from a call option by investing in the opposite site of the equation, leading to arbitrage opportunities if the option is mispriced according to the put-call parity. Any mispricings in the 'real' world would therefore be due to transaction costs, restrictions on borrowing or shorting constraints on the underlying stock as arbitrageurs would ensure the markets stay efficient.

\subsubsection{Implied Volatility}

The implied volatility stems from the general framework for European option prices of \cite{black1973pricing}. Assuming lognormal distributed returns, constant volatility of the stock and a constant relevant interest rate, and of course no transaction costs or difference between borrowing and lending rates, they present the iconic pricing formula for an european Call option:
\begin{equation}
	C_{t}(K,T)=SN\left(d_{1}\right)-Ke^{-r\left(T-t\right)}N\left(d_{2}\right)
	\label{eq:bs_call}
\end{equation}
where $$d_{1}=\frac{\ln\left(S_{0}/K\right)+\left(r-\delta+\frac{1}{2}\sigma^{2}\right)\left(T-t\right)}{\sigma\sqrt{T-t}}$$
and 
$$d_{2}=\frac{\ln\left(S_{0}/K\right)+\left(r-\delta-\frac{1}{2}\sigma^{2}\right)\left(T-t\right)}{\sigma\sqrt{T-t}}$$
with $S$ as the stock price today, $N\left(j\right)$ being the cumulative probability function evaluated in $j$ with a standard normal distribution, $K$ is the strike of the option, $r$ being the risk free rate and $T-t$ being the time to maturity. In the latter two equations, the constant term of the volatility of the underlying, $\sigma$, is present. The corresponding price for a european put option is:
\begin{equation}
	P_{t}(K,T)=Ke^{-r\left(T-t\right)}N\left(-d_{2}\right)-SN\left(-d_{1}\right)
	\label{eq:bs_put}
\end{equation}

These functions were derived by \cite{merton1973theory} with a simple assumption of investors preferring more to less. Combining the put-call parity with the price function of the european call option, I can infer the corresponding price of a put, which means that the implied volatility of a call option will be equal to that of a put option, if their strike, maturity, and underlying stock is the same. Thus any implied volatility spread different from zero will indicate market inefficiencies and/or arbitrage opportunities. This result holds regardles of the Black \& Scholes framework, as any implied volatility spread different from zero on european options represents an arbitrage opportunity in frictionless markets. 

When calculating the model-based price of an european call option, I would need to make an estimate of the volatility on the underlying stock's price. In other words, the volatility of the stock return is deterministic in regards to the option price. Therefore, given a set of option prices, I can derive the implied volatility, assuming that Equations \ref{eq:bs_call} and \ref{eq:bs_put} holds. 

As the implied volatility is deterministic, I can plug in the volatility of a stock return into Equation \ref{eq:bs_call} and get the theoretical implied price of a call option.

\subsubsection{Implied Volatility Spread}

For the american options, the pricing equation is slightly modified, as an american option always will be priced higher than their european counterpart. This is due to the early excercise premium, given that american options can be excercised at any time before maturity. The inequality of the american call option is:
\begin{equation}
	C_{t}^{a}(K,T) \geq S_{t}N\left(d_{1}\right)-Ke^{-r\left(T-t\right)}N\left(d_{2}\right)
\end{equation}
with the options being the same value only when they are far out of the money. As the call price now is an inequality and not an equality, the put-call-parity is now of course not exact either.

To model the value of an option, the volatility of the stock price is portrayed in a tree-like structure. 
Essentially, the stock return follows a Brownian motion in the \cite{black1973pricing} framework, and this process can be approximated with a binomial process using different estimations of the up- and down-movements and the risk-neutral probability of an up-movement. This binomial process is drawn to resemble a tree and shows the movement of the stock price and the corresponding value of the option in each time-increment assuming that the stock price will either move up by $u$ or down by $d$. These values are approximated by several different approaches, but most commonly is the Cox-Ross-Rubinstein (CRR) formula \citep{cox1979option}:
\begin{equation}
	\begin{cases}
		u=e^{\sigma\sqrt{\Delta t}}\\
		d=e^{-\sigma\sqrt{\Delta t}}
	\end{cases}
\end{equation}
which then can be used to calculate the risk-neutral probabilities of an up-movement of the underlying, $q$:
\begin{equation}
	q=\frac{e^{r\Delta t}-d}{u-d}
\end{equation}
As the inherent discrete time nature of binomial trees, when the time increments goes towards zero, the price of the option will approach the Black-Scholes model implied price.

It should be noted here, that the choice of the CRR model is an industry standard, and I assume that a different choice of for example the Second Order CRR or the Jarrow-Rudd model \citep{jarrow1982approximate}, will not change the results drastically, as they only show a small modification to CRR. Furthermore, I am looking at the difference in implied volatility and not the absolute value, which means that only a model with asymmetric assignment of the volatility to calls versus puts would give a significant different estimation of the implied volatility spread. In addition to the choice of model, I could also use trinomial trees instead, which would resemble a possibility of the stock to either move up, down or be constant in the next period. This would make the analysis more nuanced, but in my view not provide enough new insights to outweight the increased computational burden. Another feasible way to valuing these american options would have been to use a simulation with a simple least squares approch as presented by \ref{longstaff2001valuing}. 

Thus I can approximate the price of an option given a constant, known value of the volatility and a constant risk-free rate. This price will then be risk-neutral, and means that according to a risk-neutral view on the financial markets, an option should have that price to reflect the market's view on the future value of the option. 

Using the binomial trees, it is possible to infer a price for an american option, by assuming that at any future state, the investor will exercise the option if the future expected value is less than the (at the time) current value. According to \cite{optionmetrics} the implied volatility is calculated using the binomial tree approach of \cite{cox1979option} and letting the time increments be sufficiently small to proxy continous time. For the calculation of a call option's value at time $i$ the following function is used:
\begin{equation}
	C_{i}=\max\left\{ \begin{array}{c}
		\left(qC_{i+1}^{up}+\left(1-q\right)C_{i+1}^{down}\right)/R\\
		S_{i}-K
	\end{array}\right\} 
	\label{eq:call_price}
\end{equation}
with $q$ being the risk-neutral probability of an up-movement in the stock price and $R$ is the relevant continous risk-free rate less any relevant dividends. Using this formula iteratively from maturity and back to the present, this will yield a fair price for the option. However, when I have only the price and not the volatility, I can use this setup to find the implied volatility. Thus I iterate through different values for $\sigma$ until the estimated price of the option converges to the midpoint between best bid and offer of the option at market close every day. 

Completely analogously to the european option case, if I plug in the volatility of a stock return recursively into Equation \ref{eq:call_price}, it would result in the model-implied call price.

Traditionally when approximating the implied volalility, the researcher uses the price of the option, which incorporates all current knowledge of future dividends or stocksplits, and has the following form:
\begin{equation}
	f_{t}=S_{t}e^{R\left(T-t\right)}.
	\label{eq:forwardprice}
\end{equation}

As the put-call parity does not hold for american options, it is possible to encounter a non-zero implied volatility spread between a pair of call and put options. This spread is indicative of a relative expensive call or put. Following the notation in \cite{cremers2010deviations} the implied volatility spread is found through:
\begin{equation}
	IV_{i,t}^{call}-IV_{i,t}^{put}=VS_{i,t} 
	\label{eq:imp_vol_spread_CW}
\end{equation}
where $t$, $i$ and $j$ specifies the time, stock and combination of strike and maturity. The equaiton indicates that if the spread is positive (negative) then the call is relatively expensive (cheap) or the put is cheap (expensive). I would expect the spread to be close to zero and have a fairly small standard deviation, as I assume the markets are efficient and pricing the products close to their risk-neutral values.

\subsubsection{Risk-Neutral Pricing}

To provide a short note on the risk-neutral pricing, I will elaborate on the difference between the $\mathbb{Q}$ and $\mathbb{P}$ measure, and why it is relevant to keep their definitions in mind when doing an empirical asset pricing exercise on the basis of options and their risk-neutral inferred volatilities.

%A measure is 

As the $\mathbb{Q}$ measure is only signaling an adapted set of probabilities of future states which allows for risk-neutral pricing of any asset, it is related to the true probabilities under the $\mathbb{P}$ measure. The relation is defined in Girsanov's theorem \citep{girsanov1960transforming}. 

On a general level, the measure determines which probabilities are assigned to uncertain events. Under the $\mathbb{P}$ measure (also known as the physical/historical measure), the probabilities are the ones pertaining to the real world. This is the measure under which we observe asset dynamics.

The $\mathbb{Q}$-measure is another particular measure that is famously known from the fundamental theorem of asset pricing \citep{delbaen1994general}. It corresponds to the measure that would prevail in a world where all investors were risk-neutral, which would drive risk-premia to zero. In particular, under this measure, all asset price processes are driftless (i.e., martingales) when deflated by the value of the bank account. Another way to say this is that all assets earn the risk-free rate under the Q measure. The FTAP states that all assets can be priced as a discounted expectation under the Q measure, as in (2.1).

%The intuitive difference between the $\mathbb{Q}$ measure and $\mathbb{P}$ measure is that the first one finds the risk-neutral probabilities, meaning that the price of today is set by the market according to the risk attributed with each possible outcome, while the latter denotes the 'real' or actual probabilities of the future outcome.

As is apparent from the Girsanov theorem, $\mathbb{P}$ and $\mathbb{Q}$ measures are related through a risk premium, which is closely related to stochastic discount factors (SDF’s). On an intuitive level, one may view the $\mathbb{Q}$-measure as arising from a combination of the physical probabilities and a correction that is related to the risk-aversion of investors (and hence the SDF).

Thus I am observing mispricings under the $\mathbb{Q}$-measure in the option markets, and basing my analysis of the stock returns under a $\mathbb{P}$-measure on the magnitude of these mispricings. This results in an analysis based on the $\mathbb{P}$-measure which identifies risk premia in the stock market. 

%- On a general level, the measure determines which probabilities are assigned to uncertain events. Under the P-measure (also known as the physical/historical measure), the probabilities are the ones pertaining to the real world. This is the measure under which we observe asset dynamics.

%The Q-measure is another particular measure that is famously known from the fundamental theorem of asset pricing  (oprindelig reference: Delbaen & Schachermeyer). It corresponds to the measure that would prevail in a world where all investors were risk-neutral, which would drive risk-premia to zero. In particular, under this measure, all asset price processes are driftless (i.e., martingales) when deflated by the value of the bank account. Another way to say this is that all assets earn the risk-free rate under the Q measure. The FTAP states that all assets can be priced as a discounted expectation under the Q measure, as in (2.1).

%As is apparent from the Girsanov theorem, P and Q measures are related through a risk premium, which is closely related to stochastic discount factors (SDF’s). On an intuitive level, one may view the Q-measure as arising from a combination of the physical probabilities and a correction that is related to the risk-aversion of investors (and hence the SDF).

\subsection{Arbitrage Pricing Theory}

Arbitrage pricing theory was introduced by \cite{ross1976arbitrage} through the simple assumption of no arbitrage. He argues that investors should diversify their portfolios, so their exposure only is towards systemic risk factors and not idiosyncratic risk. And that all returns are primarily affected by a selected amount of factors. These returns are, of course, only a relative pricing mechanism, as Ross only states that the factors help determine the relative price levels of the assets. 

These factors driving the returns should be fewer than the amount of assets available for investment. Ross derives the approximate relation between a diversified investment in a variety of assets and an unknown set of factors. The derivation is based on the statement that a zero-beta net-zero investment in a set of assets also should provide zero return, so as to not provide arbitrage opportunities. A net-zero investment in a diversified portfolio with no exposure to any factors, should yield no expected payoff. 

Thus Ross builds the arguments as to all returns being approximately driven by a linear dependence on the factors. The factor model can as an extension of the APT model of Ross be written as:
\begin{equation}
	r_{t,i}=\mathbb{E}_{t-1}\left(r_{t,i}\right)+\sum_{k=1}^{K}\beta_{k,i}\cdot f_{k,t-1}+\sum_{j=1}^{J}\beta_{j,i}\cdot g_{j,t-1}+\sum_{h=1}^{H}\beta_{h,i}\cdot p_{h,t}+\epsilon_{t,i}
\end{equation}
where the returns of an asset in period $t$ is determined through its expected value, its exposure to the priced factors, $f$, the quasi factors $g$, which are not priced, and lastly the characteristics of the asset, $p$, which is only observable within the return period. Any of the relevant factors might be traded.

According to the Efficient Market Hypothesis \cite{fama1970efficient}, the investor should already incorporate any public knowledge into the pricing of the security, and the investors should demand compensation for systemic risk. This means that any idiosycratic risk factors do not demand a risk premium. The market should therefore have homogenous expectations regarding the exposures and risk factors.

%Explaining arbitrage pricing theory -  investors pricing risks in the markets - some factors are priced, some are not Efficient market hypothesis states that factors should be priced, but it should already be priced in the stock. 

%factor model : factors are traded or not traded, some are priced, some are not priced and just quasi factors, other things affecting the returns is the characteristics, which are observed within the holding period return.

\subsection{Rational Investors' Preferences}

A rational investor is characterised by preferring more to less. In the mean-variance model of \cite{markowitz1952portfolio}, the returns approximately follows a gaussian distribution, which leads to the natural conclusion that investors will make their investment decisions solely based on the expected return and the variance of the return. 

This consideration of only mean and variance means that we can implement the portfolio optimization proposed by \cite{markowitz1952portfolio} and \cite{markowitz1959efficient}. Here the covariance matrix of the returns, the expected value and the variance of the returns are assumed to remain somewhat stable. And the investor should then combine the portfolios on the efficient frontier, and identify the relevant split between the risk free asset and the optimal portfolio.

In this setup, I am not assuming risk aversion, but only that the only source of risk relevant to the investor is the variance. This is because I do not need the additional assumptions of the utility framework introduced by \cite{von2007theory}, because I am limiting the focus to the optimal portfolio formation. 

Inclusion of the utility framework would result in a choice regarding the utility function. This would model the rational investors preference under an expected utility view, and not account for several behavioural biases, which I have deemed relevant in the choice of implied volatility spreads. Namely the loss aversion which is evident as a backbone of the prospect theory introduced by \cite{kahneman2013prospect}.

\textbf{Prospect Theory} To make a brief comment on prospect theory. The choice of including the implied volatility spreads data, is partly based on the historical average being negative. I interpret this negative value to be a signal of the put options to frequently have a relatively high price, even after correcting for the intrinsic value of the option. A relatively high price of the put indicates that investors are bidding up the price of the  right to sell in the future. Thus they are willing to pay a higher price to ensure that their profits are locked in. This indicates loss aversion, as they are not behaving rationally. If all investors were rational, then the implied volaatility spread should be fairly small and close to zero, as it only would proxy the extra information of the delayed market close of the optionmarkets, and the tendency of informed investors to trade in the option markets before moving to the stock market.

%assumptions from utility 
%always preferring more to less, the investor are risk-averse, assuming that all investors therefore would only invest in risky stocks if the expected future value is higher than the risk free rate. 

%returns are normally distributed (gaussian) -> resulting in the mean-variance space being relevant -> is this true? no, so 

%mean-variance investors results in the mean-variance Efficient Set and the capital allocation line -> allocation according to optimal allocation given the ex-post analysis -> resulting in analysis of the covariance of the portfolios giving a optimal allocation according to the highest Sharpe Ratio -> allocation constraints as no shorting + a wish to be diversified or unavailability of the portfolios.

%we are not assuming risk-aversion , we are just assuming standard deviation being the relevant risk measure.

%\subsection{Prospect Theory}\label{subsection:prospect}

%loss aversion -> puts should be more expensive than calls -> rational value of the mean implied volatility spread

\subsection{A Small Note on Statistical Properties}

The statistical properties of the returns are relevant for the interpretation and validity of the results. For the regressions deployed, the returns should follow a gaussian distribution. OLS is assumed to be the efficient estimator, but the returns show some problems related to autocorrelation and heteroscedasticity. 

The autocorrelation of asset returns is a common problem, and as such I have deployed Newey-West standard errors to correct for this potential issue. This also takes care of the problems with heteroscedasticity.

Some of the models are deployed in the literature using both OLS and in a GMM framework, but to keep the simplicity of the methods, I have refrained from using GMM. The benefits from including GMM would be the robust standard errors and removing the errors-in-variables problem arising from Fama-Macbeth regressions, but this will be corrected through the use of robust standard errors and a correction by Shanken for the errors-in-variables. 

Correcting for these problems allows me to interpret and conduct inference upon the results.

The stationarity of the signal is shown in the appendix, along with the plots of the autocorrelation function of the different portfolios. 

%gaussian distributed returns -> OLS is the efficient estimator -> problems with autocorrelation and heteroscedasticity -> also in the factors ? -> solutions : robust standard errors and GMM takes this into account -> it is not used here

%NEWEY-WEST

